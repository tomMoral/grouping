\documentclass{gretsi}

\usepackage[utf8]{inputenc}
\usepackage[T1]{fontenc}
\usepackage{amsmath, amssymb}
\usepackage[english,francais]{babel}
\usepackage{algorithm}
\usepackage[noend]{algpseudocode}
\usepackage{color}
%\usepackage{enumitem}


%Gummi|065|=)
\title{Separation de sources automatique}
\author{Thomas Moreau}
\date{}

\makeatletter
\def\BState{\State\hskip-\ALG@thistlm}
\makeatother
\def\HH{\mathcal H}

\resumefrancais{Cet article introduit différentes stratégies de groupement automatique pour les composantes issues d'une analyse du spectre singulier (SSA). Cette étape, cruciale à la séparation de composantes haut niveau, permet de séparer les différentes dynamiques présentes dans une signal.}
\resumeanglais{This paper introduce automatic grouping strategies for Singular Spectrum Analysis (SSA) in an unified framework. This step is needed to be able to retrieve meaningful insight about the temporal dynamics of the series.}


\newcommand{\R}{\mathbb R}
\newcommand{\serie}[1]{(#1_i)_{1 \le i\le T}}
\newcommand{\val}[3]{(#1_1 #3 \dots #3 #1_#2)}
\newcommand{\vect}[3]{(#1_1 #3 \dots #3 #1_#2)}
\newcommand{\set}[1]{\left \{ 1, \dots, #1 \right \}}
\newcommand{\inter}{\left[0, 1\right]}


\begin{document}

\maketitle


\section{Introduction}
\label{sec:intro}

L'analyse du spectre singulier (en anglais SSA : Singular Spectral Analysis) est une technique introduite par Vautard et al \cite{vautard_ghil_89_SSA} permettant d'obtenir de manière simple une décomposition du signal en composantes de tendance, de saisonnalité (harmoniques) et de bruit. Elle est principalement utilisée en finance ou en météorologie où les séries étudiées sont souvent courtes et bruitées. De nombreuses publications ont montré la pertinence de la SSA dans ces contextes, où cette décomposition permet d'apporter de l'information sur les mécanismes produisant ces séries mais aussi de formuler des modèles de prédiction.


Néanmoins, cette technique a été rarement utilisée sur d'autres signaux, principalement à cause de son manque d'automatisation. La pertinence des décompositions obtenues dépend en effet énormément du regroupement des composantes extraites, et résoudre cette tâche de façon manuelle peut s'avérer laborieux. Cet article a pour but de présenter différentes stratégies de regroupement automatique (existantes et originales) ainsi qu'un cadre permettant leur évaluation et leur comparaison.

%Cette technique, basée sur le théorème de Karhunen-Loève, permet de décomposer un signal en une somme de composante haut niveau rattachées à sa tendance et à sa saisonnalité.
%L'analyse de la dynamique sous-jacente d'une série temporelle est très intéressante pour l'étude de phénomènes non stationnaires. .  DEJA DIT NON ?


\subsection{Rappels sur la SSA}
\label{sub:rap}
On considère un signal fini échantillonné $f = \val{f}{T}{,}$ et un entier $K \in \left \{ 1, \dots, T/2 \right \}$. La matrice de K-trajectoire $X^{(K)} \in \R^{K\times L}$ de la série $f$ est défini par:
$$ X^{(K)} = \begin{bmatrix}
	f_1 & f_2 &\dots & f_K\\
	f_2 & f_3 &\dots & f_{K+1}\\
	\dots\\
	f_{L} & f_{L+1} &\dots & f_T
\end{bmatrix}$$ avec $L = T-K$. Cette matrice contient toutes les sous séries de longueur $K$ de $f$.

Le signal peut être reconstruit naturellement à partir de cette matrice du fait de sa structure de Hankel, avec toutes les composantes sur les anti-diagonales égales. Si l'on considère une matrice $M$ n'ayant pas cette structure, le signal ayant la matrice de trajectoire la plus proche de $M$ peut être construit en moyennant le long des anti-diagonales. Cette opération permet de  passer de l'espace des matrices à celui des signaux. On notera par la suite $\HH: \R^{K\times L} \to \R^T$ l'opération d'hankelisation, qui effectue la moyenne le long des anti-diagonales. Cette opération est linéaire.



Une décomposition en valeur singulière (en anglais SVD pour Singular Value Decomposition) de la matrice de K-trajectoire $X^{(K)}$ de $f$ permet d'extraire les motifs de longueur $K$ les plus intéressants au sens de la variance. En effet, la SVD donne: 
$$X^{(K)} = U \Lambda V^T
$$ avec $U, V$ deux matrices orthogonales et $\Lambda$ une matrice diagonale ayant pour valeurs $\lambda_1\ge \dots\ge \lambda_K$. La matrice $U$ peut être vue comme un dictionnaire de motifs de taille $K$ sur lequel est projeté le signal. Ces motifs permettent de capturer au mieux la variance du signal. Les vecteurs singuliers de la SVD maximisent récursivement la covariance de $U_i$ et des sous séries de longueur $K$ de $f$.

%\paragraph{Reconstruction de composantes}


L'étape d'extraction de motif permet d'obtenir une décomposition de la matrice de trajectoire sous la forme d'une somme de matrice de rang 1:$$
X^{(K)} = \sum_{i=1}^K \lambda_i U_iV_i^T 
$$où  $U_i$.Par linéarité de l'opération d'hankelisation, on obtient une décomposition de $f = \sum_{i=1}^K f^{(i)}$ avec $f^{(i)} = \HH(\lambda_iU_iV_i^T)$ la série associée à la matrice de rang 1 $\lambda_iU_iV_i^T$.


\subsection{Regroupement des composantes}
\label{sub:grp}

Les oscillations harmoniques ou plus complexes, présentes dans la série initiale sont généralement décomposées sur plusieurs de ces composantes. Le bruit est lui aussi projeté sur plusieurs composantes. Une étape d'identification et de regroupement est alors nécessaire pour produire une décomposition intéressante. Une sélection des composantes peut être faite en se basant sur les valeurs singulières associées qui rendent compte de l'importance de la composante dans la série. Une étape de regroupement est ensuite nécessaire pour obtenir une décomposition plus simple à analyser.

Diverses approches ont été utilisées dans la littérature pour réaliser ce groupement, même si la plupart des travaux utilisant la SSA réalisent cette étape de façon empirique. En particulier, certaines heuristiques basées sur la corrélation et le spectre de fréquence des composantes ont été développées pour aider au choix des groupes \cite{GNZ_10_SSA}. Ces indicateurs, d'abord utilisés pour le regroupement manuel, ont ouvert la voie au développement de méthodes de groupement automatisé. Nous verrons dans la section \ref{sec:met} une description des principales méthodes existantes ainsi que de nouvelles stratégies, ainsi qu'un cadre unifié pour l'évaluation de ces méthodes dans la section \label{sec:eval}. PUIS RESULTATS

%Les travaux de Alexandrov et Golyandina \cite{alexandrov_05_auto} propose une première approche pour le regroupement automatique, principalement tourner à la récupération d'harmonique pure dans le signal. Dans \cite{alvarez_13_auto}, Alvarez et al. propose une seconde méthode de regroupement automatique basé sur le clustering des périodogrammes des composantes reconstruites. Cette méthode vise un regroupement plus général des composantes. Dans des travaux récents, Abalov et Gubarev \cite{abalov_14_auto} présentent deux autres stratégies pour obtenir un regroupement automatique. 

% subsection grp (end)
% section intro (end)

\section{Méthodes}
\label{sec:met}

Dans cette section, nous allons présenter plusieurs méthodes de groupement automatique pour la SSA. Une description unifiée de celles-ci permet de les comparer dans un cadre simple, comme paramètres d'une stratégie globale.

\subsection{Formulation générale}
\label{sub:form}

Les stratégies de regroupement peuvent être décrites en trois phases:
\begin{enumerate}
	\item Sélectionner des composantes intéressantes de la SSA
	\item Calculer une matrice d'adjacence entre ces composantes
	\item Regrouper les composantes adjacentes
\end{enumerate}
La première étape est effectuée en supprimant les composantes de la SVD ayant une valeur singulière $\lambda_k$ plus faible qu'un certain seuil (que l'on choisira ici adaptatif et de la forme  $\tau_l*\lambda_1$ avec $\tau_l = 0.01$). Ceci permet de filtrer les composantes de bruit en considérant qu'elles ont une part négligeable dans la variance de la série. 

La seconde étape utilise une mesure de similarité entre les composantes afin d'obtenir une matrice d'adjacence. Intuitivement, le choix de cette mesure influence fortement les résultats du regroupement. Il est par exemple parfois pertinent de se  baser sur le contenu fréquentiel des composantes afin d'obtenir un bon regroupement des composantes harmoniques.

Enfin, la troisième étape concerne la stratégie de formation des groupes à partir de la matrice d'adjacence. Cette stratégie permet de gérer la prévalence des composantes entre elles. Il est souvent intéressant de donner plus d'importance à la composante associée de plus forte valeur singulière car elle a un poids important dans la variance de la série initiale.

% subsection form (end)

\subsection{Mesures de similarité}
\label{sub:sim}

\subsubsection{Corrélation}\label{ssub:cor}
\paragraph{Corrélation (GG1)}\label{par:GG1}
    Une première mesure de similarité est basée sur la corrélation \cite{abalov_14_auto}. Deux composantes sont regroupées si la corrélation entre elles est plus haute qu'une valeur seuil $\rho_c$. Pour éviter de grouper des composantes d'importance trop éloignée, deux composantes sont groupés si le ratio de leurs valeurs singulières associées $\frac{\min(\lambda_i, \lambda_j)}{\max(\lambda_i, \lambda_j)}$ est supérieur à un seuil $\rho_1 \in \inter$. Cela évite de grouper des phénomènes d'amplitude différentes. On définit donc la matrice d'adjacence $G = (g_{i, j})_{0 \le i,j\le K}$ par:
    $$
    g_{i, j} = \begin{cases}
	    1 &\text{ si } \displaystyle\frac{\min(\lambda_i, \lambda_j)}{\max(\lambda_i, \lambda_j)} \ge \rho_1 \text{ et } \text{corr}(f^{(i)}, f^{[j)}) \ge \rho_c\\
	    0& \text{ sinon}
    \end{cases}
    $$
% paragraph GG1 (end)

\paragraph{W-Corrélation (GG3)}\label{par:GG3} 
    La w-corrélation pour deux séries $x, y$ de taille $N$ et pour une taille de fenêtre $K$ est définie par:
    $$
    \text{w-corr}(x, y) = \frac{\langle x|y\rangle_w}{\|x\|_w\|y\|_w}
    $$avec $\langle x|y\rangle_w = \sum_{i=1}^n w_i x_i y_i$, $w_i = \min(i, N-i, K)$ et $\|x\|_w = \sqrt{\langle x|x\rangle_w}$. Deux séries sont séparables par la SSA si leur w-corrélation est nulle \cite{GNZ_10_SSA}. Il est donc possible d'utiliser la valeur de la w-corrélation comme indicateur de similarité entre 2 composantes. On peut alors prendre une fonction de similarité identique à la précédente en remplaçant la corrélation par la w-corrélation. Ceci permet de filtrer les effets de bord qui peuvent apparaître dans la corrélation et ainsi d'obtenir un meilleur regroupement. Les paramètres dirigeant cette mesure de similarité sont les mêmes que ceux de (GG1) et on les fixe donc aux mêmes valeurs.
% paragraph GG3 (end)

% subsubsection cor (end)

\subsubsection{Périodogramme}\label{ssub:per}
%TODO
{\color{red} \textbf{TODO} - Citer les param dans le text explicatif} 

La fréquence des composantes obtenues peut aussi être utilisée comme un indicateur de similarité des séries. Le périodogramme d'une série $f$ de longueur $N$ est défini comme:
$$
\Pi_f(k) = \frac{1}{Z}\left|\sum_{n=1}^N f_n e^{-2i\pi n \frac{k}{N}}\right|^2
$$où $Z$ est une constante de normalisation telle que le périodogramme ait une norme unitaire. Cette estimation de la densité spectrale de puissance peut être utilisée  comme indicateur pour choisir un regroupement \cite{GNZ_10_SSA}. %TODO
{\color{red} \textbf{TODO} - REDONDANT, NON ? DONNER UN ARGUMENT}   

\paragraph{Regroupement harmonique (HG)}\label{par:HG}
    Une stratégie de regroupement efficace pour extraire les harmoniques exponentiellement modulées \cite{alexandrov_05_auto} utilise une fonction de similarité qui regroupe deux composantes successives $(j, j+1)$ si 
    $$
    \frac{1}{2}\max_{0\le k \le L/2}\left(\Pi_{U_j}(k) + \Pi_{U_{j+1}}(k)\right) \ge \rho_0 
    $$pour $\rho_0\in \left[0, 1\right]$. Cette métrique rend compte du fait que les composantes successives ont des valeurs élevées aux même fréquences dans leur périodogramme. Cette méthode ne requiert qu'un seul paramètre $\rho_0$ que l'on fixe à 0.8 pour les expérimentations. %TODO
{\color{red} \textbf{TODO} - UN PETIT COMMENTAIRE POUR LA ROUTE ? }  
% paragraph HG (end)

\paragraph{Similarité des piques (GG2)}\label{par:GG2}
    Une autre métrique de similarité basée sur les périodogrammes est définie en prenant en compte la distance $\ell-\infty$ entre les fréquences non négligeables du périodogramme \cite{abalov_14_auto}. On définit l'ensemble ordonné $F_{f^{(i)}} = \left \{ k | \Pi_{f^{(i)}}(k) \ge \rho_p \|\Pi_{f^{(i)}}\|_\infty \right \}$ et on notera $F_{f^{(i)}}(h)$ la h-eme valeur de cette ensemble. On définit alors la matrice d'adjacence:
    $$
    g_{i, j} = \begin{cases}
	    1 & \text{ if } \displaystyle\frac{\min(\lambda_i, \lambda_j)}{\max(\lambda_i, \lambda_j)} \ge \rho_1\\
	      & \text{ et } \frac{|F_{f^{(i)}}(h) - F_{f^{(i)}}(h)|}{T/2} \le \rho_2 \forall h \in \set{m}\\
	    0 & \text{ sinon}
    \end{cases}
    $$ où $m = \min\left(\left|F_{f^{(i)}}\right|, \left|F_{f^{(i)}}\right|\right)$.
    %TODO
{\color{red} \textbf{TODO} - UN PETIT COMMENTAIRE POUR LA ROUTE ?
} 
% paragraph GG2 (end)

\paragraph{K-means spectral (KM)}\label{par:KM}
    La distance entre les périodogrammes peut être utilisée avec l'algorithme des K-moyennes pour former le groupement \cite{alvarez_13_auto}. Ceci est intéressant car le choix des seuils sur les distances entre les composantes d'un groupe est adapté par les K-moyennes. Le choix du nombre de groupes $C$ est un choix critique ici. Une estimation du nombre de groupe peut être faite en estimant le rang de la matrice $\Pi = \begin{bmatrix}\Pi_{U_1}&\dots&\Pi_{U_K}\end{bmatrix}^T$. On calcul les valeurs singulières $\val{\sigma}{K}{\ge}$ de $\Pi$ et on considère $C = \arg\max_k \sigma_k \ge \rho_f \sigma_1$ avec $\rho_f \in \inter$. La similarité vaut alors 1 lorsque les composantes appartiennent au même groupe formé par l'algorithme des K-moyennes.
% paragraph KM (end)


\paragraph{Regroupement support harmonique (HGS)}\label{par:HGS}
    L'une des faiblesses de la similarité (HG) est qu'elle se concentre sur des composantes harmoniques. En effet, deux composantes sont groupées si leur périodogramme ont des pics aux mêmes endroits. Ainsi, si les composantes extraites par la SSA ont une plage de fréquence plus large, cela peut nuire au groupement car la normalisation du spectre fera passer la métrique de similarité sous le seuil de regroupement.
    
    Une idée pour palier à ce problème est de considérer le support fréquentiel des composantes et de définir le pic de fréquence comme le centre de ce support. On note le support fréquentiel de $f$ comme l'ensemble $\mathcal S_f = \left \{ k | \Pi_f(k) > \rho_1 \kappa_f \right \}$ avec $\kappa_f = \max_k \Pi_f(k)$. On note alors $\omega_f = \frac{\max \mathcal S_f + \min \mathcal S_f }{2 T}$ et $s_f = \frac{\max \mathcal S_f - \min \mathcal S_f }{2}$. On groupe alors les composantes ayant un support restreint \emph{i.e.} $s_f \le 3*\log(T)$

% paragraph HGS (end)
% subsubsection per (end)
% subsection sim (end)


\subsection{Stratégies de formation des groupes}
\label{sub:clust}


\paragraph{Méthode basique}
La stratégie de formation des groupes à partir de la matrice d'adjacence la plus simple est de considérer que toutes les composantes ont le même intérêt. On peut alors considérer que deux composantes sont dans le même groupe dès lors qu'elles sont adjacentes. C'est la méthode qui est utilisé dans tous les algorithmes qui ont été proposés jusque là.\\

%\begin{algorithm}
%\caption{Groupement uniforme}\label{alg:unif}
%Given G
%\begin{algorithmic}[1]
%\For{ i = 1 \dots K}
%\State group[i] = i
%\EndFor
%\For{ i = 1 \dots K}
%\For{ j = i+1 \dots K}
%\If{$G_{i,j} = 1$} group[j] = group[i]
%\EndIf
%\EndFor
%\EndFor
%\State \Return group
%\end{algorithmic}
%\end{algorithm}


\paragraph{Méthode hiérarchique}
Une autre possibilité est de considérer que les composantes ont une importance relative. L'importance de chaque composante peut être mesurée par la part de variance $\frac{\lambda_i}{\sum_{j=1}^n \lambda_j}$ expliquée par cette composante. On peut donc considérer qu'on ne rajoute une composante $j$ dans un groupe $I_k$ que si elle est similaire à la composante dominante du groupe \emph{ i.e. } $G_{i, j} = 1$ pour $i = \arg\max_{l \in I_k} \lambda_l$. Cela permet de donner un poids plus important lors du groupement à la composante qui contient le plus de variance du signal.

% IL FAUT PEUT ETRE ENLEVER LES 2 ALGOS?? Je SUIS PAS CERTAIN DE CE QUE CELA APPORTE AU FINAL....
%\begin{algorithm}
%\caption{Groupement ordonée}\label{alg:ord}
%Given G
%\begin{algorithmic}[1]
%\State GI $= \left \{  \right \}, I = \set{m}$
%\For{$i \in I$}
%\State remove i from $I$, $I_i = \left \{ i \right \}$
%\For{$j \in I$}
%\If{$G_{i,j} = 1$} 
%\State remove j from $I$, $I_i = I_i \cup \left \{ j \right \}$
%\EndIf
%\EndFor
%\State GI = GI$ \cup I_i$
%\EndFor
%\State \Return GI
%\end{algorithmic}
%\end{algorithm}

% subsection clust (end)
% section met (end)

\section{Méthode d'évaluation}
\label{sec:eval}

\subsection{Signaux}
\label{sub:artsig}

    Les stratégie de regroupement ont été évaluées sur des signaux artificiels générés aléatoirement. L'objectif du regroupement est de séparer la tendance des différentes composantes périodiques et du bruit. Les signaux tests sont échantillonnés à $100Hz$ selon le modèle:
    \begin{equation}\label{eq:artsig}
    f(t) = \underbrace{a_0 t^p}_{c_0} + \sum_{i=1}^K \underbrace{a_i e^{-\alpha_i t} \sin\left(f_i t + \phi_i\right)}_{c_i} + \epsilon_t
    \end{equation} avec des valeurs aléatoires tirées sous les lois $a_i \sim \mathcal U(0, 1)$, $p \sim \mathcal U(0, 5)$, $\displaystyle \phi_i \sim \mathcal U\left(\frac{-\pi}{2}, \frac{\pi}{2}\right)$, $\alpha_i \sim \mathcal U\left(0, \frac{1}{2}\right)$ et $f_i \sim \mathcal U(0, 50Hz)$.
    
    On ajoute à ce signal une composante de bruit blanc $\epsilon_t$ de variance $\sigma = s*\sigma_f$, avec $s \sim \mathcal U(0, 30dB)$ le SNR de notre signal.
%TODO
{\color{red} \textbf{TODO} - A Modifier }
% subsection artsig  (end)

\subsection{Métriques d'évaluation}
\label{sub:met}

    L'une des métriques utilisée pour évaluer la qualité d'un groupement est le coefficient de détermination $R^2$ \cite{abalov_14_auto}. Ce coefficient rend compte de l'erreur commise par rapport à la variance de la composante. Pour une série $f$ de moyenne $\overline f$ et pour un estimateur $\widehat f$, le coefficient de détermination est calculé suivant
    $$
        R^2(f, \widehat f) = 1 - \frac{\|f-\widehat f\|^2}{\|f-\overline f\|^2}
    $$Cette quantité correspond au rapport entre l'erreur d'estimation et la variance d'un signal. La métrique d'évaluation du regroupement est alors calculée en moyennant les valeurs minimales de $R^2$ pour chaque composante $c_i$ du signal d'origine et pour 
    $$
        \overline{ R^2} = \sum_{i=1}^n \min_k R^2(c_i, g_k)
    $$où les $g_k$ sont les différentes composantes obtenue lors du groupement.
    
    Cette métrique ne rend compte que du rappel de la méthode et non de la précision. Le but de la phase de groupement est d'obtenir des composantes haut niveau et de filtrer les composantes de bruit. Il est important de prendre en compte l'efficacité de la stratégie de groupement pour discerner les composantes utiles.

% subsection met (end)

\subsection{Séparabilité}
\label{sub:sep}
    L'évaluation de la stratégie de regroupement doit prendre en compte la qualité des composantes obtenues lors de la décomposition initiale du signal. En effet, si les composantes que l'on essaie d'analyser ne sont pas séparables par l'analyse de la matrice de $K$-trajectoire, les stratégies de groupement ne pourront pas améliorer cette séparation.
    
    %Il est donc important de rappeler ici les propriétés des composantes séparables. Deux signaux $f_1$ et $f_2$ sont dit séparables s'il existe $I_1, I_2 \subset \left \{ 1,\dots, K \right \}$ tel que $I_1\cap I_2 = \emptyset$ et $f_i = \mathcal H(\sum_{j \in I_i} \lambda_j U_j V_j^T)$. Ceci implique notamment l'orthogonalité de toutes les sous séries de longueur $K$ et $L$ de $f_1$ et $f_2$. Différentes illustrations de ces propriétés peuvent être trouvées dans \cite{GNZ_10_SSA}. 
       
   Pour évaluer l'amélioration apportée par le groupement pour une métrique, il est intéressant de considérer le rapport de cette métrique et de la même métrique calculer pour un groupement trivial tel que chaque groupe contienne une unique composante. $I_k = \left \{ k \right \}$ pour $k \in \set{m}$.
% subsection sep (end)
% section eval (end)

\section{Résultats}
\label{sec:res}
Le Tableau \ref{tab:1} présente la moyenne des différentes métriques pour chaque stratégie de groupement sur 2000 tirages de fonctions selon le modèle (\ref{eq:artsig}).

K fixé à T/2!! :)

\bibliographystyle{plain} 
\bibliography{bib_dl}{}
\end{document}
